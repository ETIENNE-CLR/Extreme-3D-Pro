\documentclass[french]{article}

% Config
\def\mysubtitle{Fiche de capacités}
\def\myLesson{Atelier Programmation}
\input{../includes/config.tex}

% Début du document
\begin{document}
\input{../includes/titlepage.tex}
\clearpage

\tableofcontents
\clearpage

% Corps du document
\section{Identité du périphérique}
\begin{itemize}
	\item \textbf{Marque :} Logitech
	\item \textbf{Modèle :} EXTREME 3D PRO
	\item \textbf{Type :} Joystick
	\item \textbf{Identifiants système :}
	      \begin{itemize}
		      \item USB : VID = jsp ; PID = jsp
		      \item Nom exposé au système : Extreme 3D pro
	      \end{itemize}
	\item \textbf{Firmware / version :} Je sais pas
	\item \textbf{Date du test :} \today
	\item \textbf{Testeur :} \myauthor
\end{itemize}

\section{Contrôles joystick}
\textbf{Axes analogiques}\newLine

\begin{tabular}{rcccl}
	\toprule
	Axe & ID / Nom & Plage brute & Deadzone & Remarques \\
	\midrule
	X1  & -        & -           & -        & -         \\
	Y1  & -        & -           & -        & -         \\
	Z1  & -        & -           & -        & -         \\
	X2  & -        & -           & -        & -         \\
	Y2  & -        & -           & -        & -         \\
	Z2  & -        & -           & -        & -         \\
	\bottomrule
\end{tabular} \newLine

\textbf{Boutons}\newLine

\begin{tabular}{rcl}
	\toprule
	N° & Étiquette physique & Remarques                              \\
	\midrule
	1  & Aucune étiquette   & Gachette                               \\
	2  & Bouton 2           & Bouton du pouce                        \\
	3  & Bouton 3           & Bouton bas-gauche du haut du joystick  \\
	4  & Bouton 4           & Bouton bas-droit du haut du joystick   \\
	5  & Bouton 5           & Bouton haut-gauche du haut du joystick \\
	6  & Bouton 6           & Bouton haut-droit du haut du joystick  \\
	7  & Bouton 7           & Bouton sur la base haut-gauche         \\
	8  & Bouton 8           & Bouton sur la base haut-droit          \\
	9  & Bouton 9           & Bouton sur la base milieu-gauche       \\
	10 & Bouton 10          & Bouton sur la base milieu-droit        \\
	11 & Bouton 11          & Bouton sur la base bas-gauche          \\
	12 & Bouton 12          & Bouton sur la base bas-droit           \\
	\bottomrule
\end{tabular}


\section{Comportement temporel et stabilité}
\textbf{Latence}
\begin{itemize}
	\item Latence moyenne observée (joystick) : x ms
	\item Méthode de mesure (logiciel, protocole) : yolo
\end{itemize} \newLine

\textbf{Jitter}
\begin{itemize}
	\item Variabilité de la latence (écart typique) : : x ms
	\item Niveau ressenti : faible
	\item Effet perçu joystick : bah je sais pas, ça veut dire quoi ?
\end{itemize} \newLine

\textbf{Stabilité des valeurs}
\begin{itemize}
	\item Jitter des axes au repos (bruit, petites variations) : non
\end{itemize} \newLine

\section{Limitations et particularités observées}
\textbf{Limitations techniques}
\begin{itemize}
	\item Pas de MIDI OUT physique : oui
	\item Résolution limitée de certains contrôles : je sais pas
	\item Nombre max. de boutons / axes gérés par le driver : 12 boutons et 2 joysticks
\end{itemize} \newLine

\end{document}